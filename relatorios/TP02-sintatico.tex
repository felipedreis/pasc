\documentclass[11pt]{article}

\usepackage[brazil]{babel}
\usepackage[utf8]{inputenc}
\usepackage{graphicx}
%\usepackage{multirow}
%\usepackage{subfigure}
%\usepackage{a4wide}
\usepackage{fancyhdr}
\usepackage{algorithm}
\usepackage{algorithmic}
\usepackage{tikz}
\usepackage{empheq}
\usetikzlibrary{trees}
\usepackage{amssymb,amsmath}
\usepackage{amsthm,amsfonts}
\usepackage{float}
\graphicspath{ {images/} }
\usepackage{listings}
\usepackage{color}
\usepackage{hyperref}

\pagestyle{fancy}
\renewcommand{\headrulewidth}{0.1pt}
\renewcommand{\footrulewidth}{0.1pt}



\definecolor{mygreen}{rgb}{0,0.6,0}
\definecolor{mygray}{rgb}{0.5,0.5,0.5}
\definecolor{mymauve}{rgb}{0.58,0,0.82}
\lstset{ %
  backgroundcolor=\color{white},   % choose the background color; you must add \usepackage{color} or \usepackage{xcolor}
  basicstyle=\footnotesize,        % the size of the fonts that are used for the code
  breakatwhitespace=false,         % sets if automatic breaks should only happen at whitespace
  breaklines=true,                 % sets automatic line breaking
  columns=flexible,
  captionpos=b,                    % sets the caption-position to bottom
  commentstyle=\color{mygreen},    % comment style
  %deletekeywords={...},           % if you want to delete keywords from the given language
  %escapeinside={\%*}{*)},         % if you want to add LaTeX within your code
  extendedchars=true,              % lets you use non-ASCII characters; for 8-bits encodings only, does not work with UTF-8
  frame=leftline,	               % adds a frame around the code
  keepspaces=true,                 % keeps spaces in text, useful for keeping indentation of code (possibly needs columns=flexible)
  keywordstyle=\color{blue},       % keyword style
  language=Make,                   % the language of the code
  %otherkeywords={*,...},          % if you want to add more keywords to the set
  numbers=left,                    % where to put the line-numbers; possible values are (none, left, right)
  numbersep=10pt,                  % how far the line-numbers are from the code
  numberstyle=\tiny\color{mygray}, % the style that is used for the line-numbers
  %rulecolor=\color{black},        % if not set, the frame-color may be changed on line-breaks within not-black text (e.g. comments (green here))
  showspaces=false,                % show spaces everywhere adding particular underscores; it overrides 'showstringspaces'
  showstringspaces=false,          % underline spaces within strings only
  showtabs=false,                  % show tabs within strings adding particular underscores
  stepnumber=1,                    % the step between two line-numbers. If it's 1, each line will be numbered
  stringstyle=\color{mymauve},     % string literal style
  tabsize=4,	                   % sets default tabsize to 2 spaces
  %title=\lstname                  % show the filename of files included with \lstinputlisting; also try caption instead of title
  xleftmargin=1cm
}




\lhead{Analisador Léxico e Tabela de Símbolos}
\rhead{\thepage} 
\lfoot{ Felipe Duarte, Gustavo Borba, Juan Lopes }
\rfoot{ \today }
\cfoot{}


\begin{document}
\thispagestyle{empty}

\begin{center}
\begin{minipage}[l]{10cm}{
\center
Compiladores \\
2016/01 \\
}\end{minipage}
 \vfill
 \begin{minipage}[l]{11cm}{
   \begin{center}
   \Large{Trabalho Prático 01: \\ Analisador Léxico e Tabela de Símbolos}
   \end{center}
}\end{minipage}
\end{center}
 \vspace*{8cm}
 \begin{center}
 \begin{minipage}[l]{10cm}{
 \center Felipe Duarte, Gustavo  Borba, Juan Lopes\\
 Belo Horizonte, \today \\
 }
 \end{minipage}
 \end{center}

\newpage
\thispagestyle{empty}
\tableofcontents

\newpage
\clearpage
\setcounter{page}{1}

\section{Introdução}
	O programa pas-c é um projeto desenvolvido na aula de Compiladores do Curso de Engenharia da Computação do CEFET-MG. 
	Pas-c é a linguagem homônima, definida para esse compilador, cuja semântica dos comandos e expressões é a tradicional 
	de linguagens como Pascal e C (daí seu nome).
	
	\subsection{Modificações na Gramática da Linguagem}
	\label{gramatica}

		Na primeira parte do trabalho foi apresentada a gramática original da linguagem proposta em sala de aula.
		A fim de tornar a implementação de um parser descendente recursivo possível, 
		foram necessárias modificações nessa gramática, que são descritas na presente seção.
	
		Eliminamos recursão à esquerda das produções \textit{simple-expr} e \textit{term}, e retiramos produções
		não significantes para as construções gramaticais, i.e., aquelas produções que geram apenas um 
		símbolo não-terminal.
		
		Para eliminar a recursão a esquerda das produções de expressões aritiméticas, trocamos a ordem que elas devem
		ser interpretadas, e.g., na gramatica original consta \textit{term ::= term mulop factor-a} o que semanticamente
		é o mesmo que \textit{term ::= factor-a mulop term} graças a associatividade do operador \textit{mulop}.
		
		A gramática resultante é apresentada em \autoref{lst:gramaticaResultante}.
		\newpage
\begin{lstlisting}[label={lst:gramaticaResultante}, caption={Gramática alterada para implementação do parser recursivo}]
program ::= [ var decl-list] begin stmt-list end
decl-list ::= decl ";" { decl ";"}
decl ::= ident-list is type
ident-list ::= identifier {"," identifier}
type ::= int | string
stmt-list ::= stmt ";" { stmt ";"}
stmt ::= assign-stmt | if-stmt | do-stmt | read-stmt | write-stmt
assign-stmt ::= identifier ":=" simple_expr
if-stmt ::= if expression then stmt-list end 
			| if expression then stmt-list else stmt-list end
do-stmt ::= do stmt-list while expression
read-stmt ::= in "(" identifier ")"
write-stmt ::= out "(" simple-expr ")"
expression ::= simple-expr | simple-expr relop simple-expr
simple-expr ::= term | term addop simple-expr
term ::= factor-a | factor-a mulop term
fator-a ::= factor | not factor | "-" factor
factor ::= identifier | constant | "(" expression ")"
relop ::= "=" | ">" | ">=" | "<" | "<=" | "<>"
addop ::= "+" | "-" | or
mulop ::= "*" | "/" | and
constant ::= integer_const | literal
integer_const ::= nozero {digit} | "0"
literal ::= " {" {caractere} "}"
identifier ::= (letter) {letter | digit } 
				| "_" (letter | digit ) {letter | digit }
letter ::= [A-Za-z]
digit ::= [0-9]
nozero ::= [1-9]
caractere ::= um dos 256 caracteres do conjunto ASCII, 
				exceto "{", "}" e quebra de linha
\end{lstlisting}

\newpage
\section{Uso do compilador}

	
	\subsection{Compilando o Compilador Pas-c}

		Para compilar o projeto (neste momento, apenas como analizador sintático), basta ter um compilador G++ versão 4.8 
		ou superior com cmake ou make instalado em sua máquina.
		
		Na pasta do projeto, execute o comando:
\begin{lstlisting}[language=C++]
make
\end{lstlisting} 
		
		Esse comando deverá gerar a seguinte seqûencia de comandos:
\begin{lstlisting}[language=C++]
g++ -Iinclude -c -g  src/main.cpp -o src/main.o
g++ -Iinclude -c -g  src/Scanner.cpp -o src/Scanner.o
g++ -Iinclude -c -g  src/Scope.cpp -o src/Scope.o
g++ -Iinclude -c -g  src/Symbol.cpp -o src/Symbol.o
g++ -Iinclude -c -g  src/Syntax.cpp -o src/Syntax.o
g++ -Iinclude -c -g  src/TestCase.cpp -o src/TestCase.o
g++ -Iinclude -c -g  src/Token.cpp -o src/Token.o
g++ -Iinclude -c -g  src/TokenType.cpp -o src/TokenType.o
g++  src/main.o src/Scanner.o src/Scope.o src/Symbol.o src/Syntax.o src/TestCase.o src/Token.o src/TokenType.o -o pasc
\end{lstlisting}

	Utilizando a diretiva \textit{test} do comando make é possível visualizar o resultado dos testes unitários.
	
	\subsection{Compilando um programa em Pas-c}
	
		Para executar o compilador, basta executar o comando:
		
		\begin{verbatim}
			pasc caminho_para_seu_codigo.pasc
		\end{verbatim}
	

\newpage
\section{A Implementação do Compilador}
	
	\subsection{A abordagem utilizada na implementação}
		
		O projeto foi desenvolvido utilizando um analisador sintático recursivo descendente,
		em que para cada produção gramatical existe um método que espera um determinado tipo de 
		token na entrada e toma a decisão de que método seguir baseada nesste token.
		
	\subsection{A Estrutura do Projeto}
		
		O projeto segue a estrutura básica de todo projeto em C++, contendo uma pasta \textbf{include} e, dentro desta, separados por módulos, os headers (.h) com as declarações das classes. A pasta \textbf{src} contém a implementação das classes, bem como a função principal \textbf{main}:
		
		\begin{verbatim}
			include/ -- Pasta para os headers
			include/frontend -- pasta contendo as definições para os analisadores.
			include/backend -- ainda por fazer.
			include/test -- Pasta com os headers referentes aos testes unitários
			src/ -- Implementação das classes
			tests/ -- Pasta com a implementação dos testes unitários 
		\end{verbatim}
		
		
		
	\subsection{Principais Classes da Aplicação}
		
		
		\begin{itemize}
			
			\item \textbf{Token: }A classe Token descreve a unidade lógica mais básica do compilador. 
			Ela apresenta apenas um valor, em formato de cadeia de caracteres, e seu \textbf{Tipo}, 
			a ser elucidado no próximo item. 
			
			\item \textbf{TokenType: } TokenType trata-se apenas de um enum, que define, 
			através de um bitmap, a lista de tokens reconhecidos pelo compilador, bem como um tipo UNKNOWN, 
			para todo e qualquer token que não corresponder a nenhuma definição da linguagem.
			
			\item \textbf{Scanner: } O analisador léxico está concentrado basicamente na classe Scanner. 
			Essa classe contém os métodos \textbf{getNumerical}, \textbf{getLiteral} e \textbf{getOperator}, 
			que são responsáveis por captar os tokens, bem como o método \textbf{getString}. 
			Este último é necessário por que uma vez que se abre um caracter delimitador de string, 
			espacos em brancos passam a ser caracteres significativos. 
			Também estão definidos nessa classe os contadores de linhas e colunas do código-fonte.
			
			\item \textbf{Scope: } É a classe que definie o escopo de uma variável. 
			Cada instância de Scope contém uma tabela com os símbolos daquele escopo e um ponteiro para seu "escopo pai".
			
			\item \textbf{Syntax: } É a classe que implementa o parser recursivo descendente. 
			ela comunica com o analisador léxico (\textbf{Scanner}), solicitando o próximo Token lido do arquivo de entrada.
			Os métodos privados \textbf{findX()}, onde X é uma produção da gramática, sabem como interpretar cada produção
			esperando encontrar determinados tokens na entrada ou outras produções validas. Quando o analisador sintático 
			encontra algum token inesperado na entrada, imediatamente uma exceção é lançada. 
			Ele também é responsável por instalar os identificadores na tabela de símbolos \textbf{Scope}. 
			Excessões também serão lançadas  ao encontrar uma declaração duplicada de variável, ou o uso de um identificador
			não instalado.  
		
		\end{itemize}
		
	\subsection{Excessões tratadas}
	
		\begin{itemize}
			
			\item \textbf{MalformedIdentifier} Lançada quando o \textbf{Scanner} encontra identificadores com caracteres inválidos
			ou cujo tamanho excede o limite máximo de quinze caracteres.
			
			\item \textbf{SymbolAlreadyInstalled} Lançada quando o \textbf{Syntax} tenta instalar um simbolo na tabela que já
			fora previamente instalado
			
			\item \textbf{SymbolNotFound} Lançada quando um identificador não é encontrado na tabela de símbolos
			
			\item \textbf{SyntaxError} Lançada quando um token inesperado é encontrado na entrada
			
			\item \textbf{UnknownOperator} Lançada quando um caractere não é identificado como operador válido da linguagem
		\end{itemize}
		
	\subsection{Testes unitários}
	Uma micro-biblioteca de testes unitários baseada em macros foi desenvolvida. 
	As principais macros no arquivo \textit{test/Assertions.h} são 
	\textbf{ASSERT\_TRUE}, \textbf{ASSERT\_FALSE}, \textbf{SHOULD\_PASS}, e \textbf{SHOULD\_FAIL}.
	Os testes unitários são executados antes da aplicação quando ela for compilada usando o comando \textbf{make test}.
  
\newpage
\section{Resultados dos testes especificados}

	Os testes aqui realizados não exibem a tabela de símbolos: Esta é responsabilidade do analizador Sintático, que será realizado na próxima etapa.
	Nosso projeto tem uma classe Scope, justamente para compreender essa situação, que ainda não está preparada.


	\subsection{Teste 1}
	
		Código fonte testado:
		\lstinputlisting[language=Pascal]{../tests/test1.psc}
			
		Saída encontrada:
		\lstinputlisting{../tests/output/test1.out}
	
	
	\subsection{Teste 2}
		
		Código fonte testado:
		\lstinputlisting[language=Pascal]{../tests/test2.psc}
			
		Saída encontrada:
		\lstinputlisting{../tests/output/test2.out}


	\subsection{Teste 3}
	
		Código fonte testado:
		\lstinputlisting[language=Pascal]{../tests/test3.psc}
			
		Saída encontrada:
		\lstinputlisting{../tests/output/test3.out}


	\subsection{Teste 4}
	
		Código fonte testado:
		\lstinputlisting[language=Pascal]{../tests/test4.psc}
			
		Saída encontrada:
		\lstinputlisting{../tests/output/test4.out}
		
	
	\subsection{Teste 5}
	
		Código fonte testado:
		\lstinputlisting[language=Pascal]{../tests/test5.psc}
			
		Saída encontrada:
		\lstinputlisting{../tests/output/test5.out}

	\newpage
	\subsection{Teste 6}
	
		Código fonte testado:
		\lstinputlisting[language=Pascal]{../tests/test6.psc}
			
		Saída encontrada:
		\lstinputlisting{../tests/output/test6.out}

	\newpage
	\subsection{Teste 7}
	
		Código fonte testado:
		\lstinputlisting[language=Pascal]{../tests/test7.psc}
			
		Saída encontrada:
		\lstinputlisting{../tests/output/test7.out}


	\subsection{Teste 8}
	
		Código fonte testado:
		\lstinputlisting[language=Pascal]{../tests/test8.psc}
		
		Saída encontrada:
		\lstinputlisting{../tests/output/test8.out}



\end{document} 